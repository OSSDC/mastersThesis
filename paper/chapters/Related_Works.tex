\chapter{Related Work}

There are several different ways to implement a stereo vision system. Many stereo vision systems are implemented on field-programmable gateway arrays (FPGAs). FPGAs allow for parallelization when processing images. Systems that use FPGAs generally can achieve a high frames per second with a decent or good image quality, but most of these systems are expensive. 

FPGA Design and Implementation of a Real-Time Stereo Vision System~\cite{alteraStratixIVPaper} uses an Altera Stratix IV GX DE4 FPGA board to process the right and left images that come from the cameras that were attached to it.~\cite{alteraStratixIVPaper} uses the Sum of Absolute Differences (SAD) algorithm to compute distances. This system allows for real time speeds, up to 15 frames per second at an image resolution of 1280x1024. However, the Altera Stratix IV GX DE4 FPGA board costs over \$4,000~\cite{alteraStratixIVBoard}, which makes the system impractical for non-high budget projects.

Improved Real-time Correlation-based FPGA Stereo Vision System~\cite{xilinxVirtex5Paper} uses a Xilinx Virtex-5 board to process images.~\cite{xilinxVirtex5Paper} uses a correlation-based algorithm, which is based on the Census Transform, to obtain the depth in images. The algorithm is fast, but there are some inherent weaknesses to it. This system can run at 70 frames per second for images at a resolution of 512x512. Unfortunately, the Xilinx Virtex-5 board costs more than \$1,000~\cite{xilinxVirtex5Board}, which is still expensive.

Low-Cost Stereo Vision on an FPGA~\cite{lowCost1000} uses a Xilinx Spartan-3 XC3S2000 board.~\cite{lowCost1000} uses the Census Transform algorithm for image processing. This allows images with a resolution of 320x240 to be processed at 150 frames per second. The total hardware for the low-cost prototype used in~\cite{lowCost1000} costs just over \$1,000, which is a bit too pricy for a lot of projects.

An Embedded Stereo Vision Module For Industrial Vehicles Automation~\cite{xilinxSpartan3APaper} uses a Xilinx Spartan-3A-DSP FGPA board.~\cite{xilinxSpartan3APaper} uses an Extended Kalman Filter (EKF) based visual simultaneous localization and mapping (SLAM) algorithm. The accuracy of this system directly varied with speed and distance of detected object. The Xilinx Spartan-3A-DSP FGPA board is around \$600~\cite{xilinxSpartan3ABoard}, which is less expensive than the other systems presented so far.

Several commercial stereo vision systems exist presently~\cite{xilinxSpartan3APaper}. Most of them are quite capable of producing good quality depth maps of their surroundings. However, the cost of these products can be relatively expensive, especially from a club or hobbyist standpoint. The Bumblebee2~\cite{bumblebee} from Point Gray is able to produce disparity maps at a rate of 48 frames per second for an image size of 640x480, but it costs somewhere around \$1,000 or so. Having been involved with the Cal Poly Robotics Club for 6 years and seen the budgets each project in the club usually gets, \$1,000 would be most of a project's budget for the year. That kind of money could be better spent elsewhere projects.

During the course of this thesis, a stereo vision surveillance application paper~\cite{surveillance} was published that used the Digilent Atlys board~\cite{atlysBoard}. A stereo camera module, VmodCAM~\cite{vmodcam}, can be purchased with the Atlys board and was also used. The Atlys board is relatively inexpensive, at least by the standards presented thus far, at \$230 for academic use. With the VmodCAM included, the price goes up to around \$350, which is still significantly cheaper than the other FPGA boards presented from other papers. The costs and capacity of the board are why the Atlys board was selected for use in this thesis (the selection was independent of the surveillance paper). The surveillance paper used the AD Census Transform to calculate distance. Their board output the disparity map data over HDMI to a monitor. The output image is rather noisy, but it is very easy for a human to understand what is in the image, which is its intended purpose.