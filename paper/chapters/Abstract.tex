Stereo vision uses two adjacent cameras to create a 3D image of the world. A depth map can be created by comparing the offset of the corresponding pixels from the two cameras. However, for real-time stereo vision, the image data needs to be processed at a reasonable frame rate. Real-time stereo vision allows for mobile robots to more easily navigate terrain and interact with objects by providing both the images from the cameras and the depth of the objects. Fortunately, the image processing can parallelized in order to increase the processing speed. Field Programmable Gateway Arrays (FPGAs) are highly parallelizable and lend themselves well to this problem.

This thesis presents a stereo vision module which uses the Sum of Absolute Differences (SAD) algorithm. The SAD algorithm uses regions of pixels called windows to compare pixels to find matching pairs for determining depth. Two  implementations are presented that utilize the SAD algorithm in differently. The first implementation uses a 9x9 window for comparison and is able to process 4 pixels simultaneously. The second implementation uses a 7x7 window and processes 2 pixels simultaneously, but parallelizes the SAD algorithm for faster processing. The 9x9 implementation creates a better depth image that has less noise, but the 7x7 implementation is shown to process images at a higher frame rate. It has been shown through simulation that the 9x9 and 7x7 are able to process an image size of 640x480 at a frame rate of 11.26 and 16.23, respectively.

%Stereo vision systems allow for depth to be associated with objects within images. However, for real-time stereo vision, a lot of data needs to be processed relatively quickly. Real-time stereo vision allows for mobile robots to more easily navigate terrain and interact with objects by providing both the images from the cameras and the depth of the objects in those images. Fortunately, the image processing can parallelized in order to increase the processing speed. Field Programmable Gateway Arrays (FPGAs) are highly parallelizable and lend themselves well to this problem.